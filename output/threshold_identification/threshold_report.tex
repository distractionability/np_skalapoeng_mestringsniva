% Options for packages loaded elsewhere
\PassOptionsToPackage{unicode}{hyperref}
\PassOptionsToPackage{hyphens}{url}
\documentclass[
  10pt,
]{article}
\usepackage{xcolor}
\usepackage[margin=0.75in]{geometry}
\usepackage{amsmath,amssymb}
\setcounter{secnumdepth}{-\maxdimen} % remove section numbering
\usepackage{iftex}
\ifPDFTeX
  \usepackage[T1]{fontenc}
  \usepackage[utf8]{inputenc}
  \usepackage{textcomp} % provide euro and other symbols
\else % if luatex or xetex
  \usepackage{unicode-math} % this also loads fontspec
  \defaultfontfeatures{Scale=MatchLowercase}
  \defaultfontfeatures[\rmfamily]{Ligatures=TeX,Scale=1}
\fi
\usepackage{lmodern}
\ifPDFTeX\else
  % xetex/luatex font selection
\fi
% Use upquote if available, for straight quotes in verbatim environments
\IfFileExists{upquote.sty}{\usepackage{upquote}}{}
\IfFileExists{microtype.sty}{% use microtype if available
  \usepackage[]{microtype}
  \UseMicrotypeSet[protrusion]{basicmath} % disable protrusion for tt fonts
}{}
\makeatletter
\@ifundefined{KOMAClassName}{% if non-KOMA class
  \IfFileExists{parskip.sty}{%
    \usepackage{parskip}
  }{% else
    \setlength{\parindent}{0pt}
    \setlength{\parskip}{6pt plus 2pt minus 1pt}}
}{% if KOMA class
  \KOMAoptions{parskip=half}}
\makeatother
\usepackage{graphicx}
\makeatletter
\newsavebox\pandoc@box
\newcommand*\pandocbounded[1]{% scales image to fit in text height/width
  \sbox\pandoc@box{#1}%
  \Gscale@div\@tempa{\textheight}{\dimexpr\ht\pandoc@box+\dp\pandoc@box\relax}%
  \Gscale@div\@tempb{\linewidth}{\wd\pandoc@box}%
  \ifdim\@tempb\p@<\@tempa\p@\let\@tempa\@tempb\fi% select the smaller of both
  \ifdim\@tempa\p@<\p@\scalebox{\@tempa}{\usebox\pandoc@box}%
  \else\usebox{\pandoc@box}%
  \fi%
}
% Set default figure placement to htbp
\def\fps@figure{htbp}
\makeatother
\setlength{\emergencystretch}{3em} % prevent overfull lines
\providecommand{\tightlist}{%
  \setlength{\itemsep}{0pt}\setlength{\parskip}{0pt}}
\usepackage{booktabs}
\usepackage{longtable}
\usepackage{array}
\usepackage{booktabs}
\usepackage{longtable}
\usepackage{array}
\usepackage{multirow}
\usepackage{wrapfig}
\usepackage{float}
\usepackage{colortbl}
\usepackage{pdflscape}
\usepackage{tabu}
\usepackage{threeparttable}
\usepackage{threeparttablex}
\usepackage[normalem]{ulem}
\usepackage{makecell}
\usepackage{xcolor}
\usepackage{bookmark}
\IfFileExists{xurl.sty}{\usepackage{xurl}}{} % add URL line breaks if available
\urlstyle{same}
\hypersetup{
  pdftitle={Threshold Identification Report},
  hidelinks,
  pdfcreator={LaTeX via pandoc}}

\title{Threshold Identification Report}
\usepackage{etoolbox}
\makeatletter
\providecommand{\subtitle}[1]{% add subtitle to \maketitle
  \apptocmd{\@title}{\par {\large #1 \par}}{}{}
}
\makeatother
\subtitle{Cutpoints for mestringsnivå classification (2014-2021)}
\author{}
\date{\vspace{-2.5em}}

\begin{document}
\maketitle

{
\setcounter{tocdepth}{3}
\tableofcontents
}
\section{Overview}\label{overview}

This report documents the cutpoints (thresholds) used to map skalapoeng
(scale scores) to mestringsnivå (mastery levels) in the Norwegian
national test data for the period 2014-2021.

\section{Methodology}\label{methodology}

\subsection{Threshold Setting Process (per Udir
documentation)}\label{threshold-setting-process-per-udir-documentation}

According to Udir, thresholds were set in the first test year to achieve
the following percentile distributions:

\begin{itemize}
\tightlist
\item
  \textbf{Grade 5}: 25\% -- 50\% -- 25\% (Levels 1, 2, 3)
\item
  \textbf{Grade 8}: 10\% -- 20\% -- 40\% -- 20\% -- 10\% (Levels 1, 2,
  3, 4, 5)
\end{itemize}

\begin{longtable}[t]{cccl}
\toprule
Grade & Threshold & Lower → Upper & Target Pctl\\
\midrule
5 & 1 & 1 → 2 & 25th\\
5 & 2 & 2 → 3 & 75th\\
\midrule\\
8 & 1 & 1 → 2 & 10th\\
8 & 2 & 2 → 3 & 30th\\
8 & 3 & 3 → 4 & 70th\\
\addlinespace
8 & 4 & 4 → 5 & 90th\\
\bottomrule
\end{longtable}

\subsection{Empirical Identification}\label{empirical-identification}

For each year × subject × grade × threshold, we identify:

\begin{enumerate}
\def\labelenumi{\arabic{enumi}.}
\tightlist
\item
  \textbf{lower\_max}: Maximum score among students at the lower level
\item
  \textbf{upper\_min}: Minimum score among students at the higher level
\end{enumerate}

The classification rule is:
\texttt{score\ \textgreater{}=\ threshold\ →\ higher\ level}

With 3-decimal precision scores, we can identify thresholds exactly.

\newpage

\section{Percentile Analysis: First Year
Thresholds}\label{percentile-analysis-first-year-thresholds}

These tables compare target vs actual percentiles for the first year
(2014).

\subsection{Grade 5}\label{grade-5}

\subsubsection{English (Engelsk)}\label{english-engelsk}

\begin{longtable}[t]{cccccc}
\toprule
Lower nivå & Upper nivå & Target Pctl & Score at Target & Actual Cutpoint & Actual Pctl\\
\midrule
1 & 2 & 25\% & 42.55 & 42.5 & 24.8\%\\
2 & 3 & 75\% & 56.51 & 56.5 & 75.0\%\\
\bottomrule
\end{longtable}

\subsubsection{Mathematics (Regning)}\label{mathematics-regning}

\begin{longtable}[t]{cccccc}
\toprule
Lower nivå & Upper nivå & Target Pctl & Score at Target & Actual Cutpoint & Actual Pctl\\
\midrule
1 & 2 & 25\% & 42.61 & 42.5 & 24.6\%\\
2 & 3 & 75\% & 56.83 & 56.5 & 74.1\%\\
\bottomrule
\end{longtable}

\subsubsection{Reading (Lesing)}\label{reading-lesing}

\begin{longtable}[t]{cccccc}
\toprule
Lower nivå & Upper nivå & Target Pctl & Score at Target & Actual Cutpoint & Actual Pctl\\
\midrule
1 & 2 & 25\% & 43.06 & 42.5 & 23.2\%\\
2 & 3 & 75\% & 56.87 & 56.5 & 73.8\%\\
\bottomrule
\end{longtable}

\subsection{Grade 8}\label{grade-8}

\subsubsection{English (Engelsk)}\label{english-engelsk-1}

\begin{longtable}[t]{cccccc}
\toprule
Lower nivå & Upper nivå & Target Pctl & Score at Target & Actual Cutpoint & Actual Pctl\\
\midrule
1 & 2 & 10\% & 36.89 & 36.5 & 9.2\%\\
2 & 3 & 30\% & 44.00 & 43.5 & 28.4\%\\
3 & 4 & 70\% & 55.96 & 55.5 & 68.6\%\\
4 & 5 & 90\% & 63.24 & 62.5 & 88.5\%\\
\bottomrule
\end{longtable}

\subsubsection{Mathematics (Regning)}\label{mathematics-regning-1}

\begin{longtable}[t]{cccccc}
\toprule
Lower nivå & Upper nivå & Target Pctl & Score at Target & Actual Cutpoint & Actual Pctl\\
\midrule
1 & 2 & 10\% & 36.88 & 37 & 10.3\%\\
2 & 3 & 30\% & 44.42 & 45 & 32.0\%\\
3 & 4 & 70\% & 55.31 & 55 & 68.9\%\\
4 & 5 & 90\% & 63.11 & 63 & 89.8\%\\
\bottomrule
\end{longtable}

\subsubsection{Reading (Lesing)}\label{reading-lesing-1}

\begin{longtable}[t]{cccccc}
\toprule
Lower nivå & Upper nivå & Target Pctl & Score at Target & Actual Cutpoint & Actual Pctl\\
\midrule
1 & 2 & 10\% & 37.23 & 36.5 & 8.7\%\\
2 & 3 & 30\% & 44.54 & 43.5 & 26.5\%\\
3 & 4 & 70\% & 55.34 & 54.5 & 67.2\%\\
4 & 5 & 90\% & 63.13 & 62.5 & 88.9\%\\
\bottomrule
\end{longtable}

\newpage

\section{Cutpoint Tables by Subject}\label{cutpoint-tables-by-subject}

\subsection{Grade 5}\label{grade-5-1}

\subsubsection{English (Engelsk)}\label{english-engelsk-2}

\begin{longtable}[t]{ccc}
\toprule
Lower nivå & Upper nivå & 2014-2021\\
\midrule
1 & 2 & 42.5\\
2 & 3 & 56.5\\
\bottomrule
\end{longtable}

\subsubsection{Mathematics (Regning)}\label{mathematics-regning-2}

\begin{longtable}[t]{ccc}
\toprule
Lower nivå & Upper nivå & 2014-2021\\
\midrule
1 & 2 & 42.5\\
2 & 3 & 56.5\\
\bottomrule
\end{longtable}

\subsubsection{Reading (Lesing)}\label{reading-lesing-2}

\begin{longtable}[t]{cclc}
\toprule
Lower nivå & Upper nivå & Years & Cutpoint\\
\midrule
1 & 2 & 2014-2021 & 42.5\\
2 & 3 & 2014, 2015 & 56.5\\
2 & 3 & 2016-2021 & 57.5\\
\bottomrule
\end{longtable}

\subsection{Grade 8}\label{grade-8-1}

\subsubsection{English (Engelsk)}\label{english-engelsk-3}

\begin{longtable}[t]{ccc}
\toprule
Lower nivå & Upper nivå & 2014-2021\\
\midrule
1 & 2 & 36.5\\
2 & 3 & 43.5\\
3 & 4 & 55.5\\
4 & 5 & 62.5\\
\bottomrule
\end{longtable}

\subsubsection{Mathematics (Regning)}\label{mathematics-regning-3}

\begin{longtable}[t]{cclc}
\toprule
Lower nivå & Upper nivå & Years & Cutpoint\\
\midrule
1 & 2 & 2014 & 37.0\\
1 & 2 & 2015-2021 & 36.5\\
2 & 3 & 2014 & 45.0\\
2 & 3 & 2015-2021 & 44.5\\
3 & 4 & 2014 & 55.0\\
\addlinespace
3 & 4 & 2015-2021 & 54.5\\
4 & 5 & 2014 & 63.0\\
4 & 5 & 2015-2021 & 62.5\\
\bottomrule
\end{longtable}

\subsubsection{Reading (Lesing)}\label{reading-lesing-3}

\begin{longtable}[t]{cclc}
\toprule
Lower nivå & Upper nivå & Years & Cutpoint\\
\midrule
1 & 2 & 2014, 2015 & 36.5\\
1 & 2 & 2016-2021 & 37.5\\
2 & 3 & 2014, 2016-2021 & 43.5\\
2 & 3 & 2015 & 44.5\\
3 & 4 & 2014-2021 & 54.5\\
\addlinespace
4 & 5 & 2014-2021 & 62.5\\
\bottomrule
\end{longtable}

\newpage

\section{Detailed Year-by-Year
Thresholds}\label{detailed-year-by-year-thresholds}

For reference, exact upper\_min values. Small variations (e.g., 42.500
vs 42.501) reflect measurement precision, not threshold differences.

\subsection{Grade 5}\label{grade-5-2}

\subsubsection{English (Engelsk)}\label{english-engelsk-4}

\begingroup\fontsize{9}{11}\selectfont

\begin{longtable}[t]{ccrrrrrrrr}
\toprule
Lower & Upper & 2014 & 2015 & 2016 & 2017 & 2018 & 2019 & 2020 & 2021\\
\midrule
1 & 2 & 42.501 & 42.500 & 42.500 & 42.500 & 42.500 & 42.500 & 42.500 & 42.500\\
2 & 3 & 56.500 & 56.500 & 56.500 & 56.500 & 56.500 & 56.500 & 56.500 & 56.500\\
\bottomrule
\end{longtable}
\endgroup{}

\subsubsection{Mathematics (Regning)}\label{mathematics-regning-4}

\begingroup\fontsize{9}{11}\selectfont

\begin{longtable}[t]{ccrrrrrrrr}
\toprule
Lower & Upper & 2014 & 2015 & 2016 & 2017 & 2018 & 2019 & 2020 & 2021\\
\midrule
1 & 2 & 42.500 & 42.500 & 42.500 & 42.500 & 42.501 & 42.500 & 42.500 & 42.500\\
2 & 3 & 56.500 & 56.500 & 56.502 & 56.500 & 56.501 & 56.500 & 56.500 & 56.500\\
\bottomrule
\end{longtable}
\endgroup{}

\subsubsection{Reading (Lesing)}\label{reading-lesing-4}

\begingroup\fontsize{9}{11}\selectfont

\begin{longtable}[t]{ccrrrrrrrr}
\toprule
Lower & Upper & 2014 & 2015 & 2016 & 2017 & 2018 & 2019 & 2020 & 2021\\
\midrule
1 & 2 & 42.500 & 42.500 & 42.501 & 42.500 & 42.500 & 42.500 & 42.501 & 42.500\\
2 & 3 & 56.500 & 56.500 & 57.500 & 57.500 & 57.500 & 57.500 & 57.500 & 57.501\\
\bottomrule
\end{longtable}
\endgroup{}

\subsection{Grade 8}\label{grade-8-2}

\subsubsection{English (Engelsk)}\label{english-engelsk-5}

\begingroup\fontsize{9}{11}\selectfont

\begin{longtable}[t]{ccrrrrrrrr}
\toprule
Lower & Upper & 2014 & 2015 & 2016 & 2017 & 2018 & 2019 & 2020 & 2021\\
\midrule
1 & 2 & 36.500 & 36.500 & 36.500 & 36.500 & 36.500 & 36.501 & 36.500 & 36.500\\
2 & 3 & 43.500 & 43.500 & 43.500 & 43.501 & 43.500 & 43.500 & 43.500 & 43.500\\
3 & 4 & 55.500 & 55.500 & 55.500 & 55.500 & 55.500 & 55.500 & 55.500 & 55.500\\
4 & 5 & 62.501 & 62.501 & 62.500 & 62.501 & 62.500 & 62.500 & 62.500 & 62.501\\
\bottomrule
\end{longtable}
\endgroup{}

\subsubsection{Mathematics (Regning)}\label{mathematics-regning-5}

\begingroup\fontsize{9}{11}\selectfont

\begin{longtable}[t]{ccrrrrrrrr}
\toprule
Lower & Upper & 2014 & 2015 & 2016 & 2017 & 2018 & 2019 & 2020 & 2021\\
\midrule
1 & 2 & 37.000 & 36.500 & 36.500 & 36.500 & 36.500 & 36.501 & 36.500 & 36.501\\
2 & 3 & 45.000 & 44.500 & 44.500 & 44.500 & 44.501 & 44.500 & 44.500 & 44.500\\
3 & 4 & 55.000 & 54.500 & 54.500 & 54.500 & 54.500 & 54.500 & 54.500 & 54.501\\
4 & 5 & 63.000 & 62.500 & 62.500 & 62.501 & 62.501 & 62.500 & 62.501 & 62.501\\
\bottomrule
\end{longtable}
\endgroup{}

\subsubsection{Reading (Lesing)}\label{reading-lesing-5}

\begingroup\fontsize{9}{11}\selectfont

\begin{longtable}[t]{ccrrrrrrrr}
\toprule
Lower & Upper & 2014 & 2015 & 2016 & 2017 & 2018 & 2019 & 2020 & 2021\\
\midrule
1 & 2 & 36.500 & 36.503 & 37.500 & 37.500 & 37.500 & 37.500 & 37.501 & 37.500\\
2 & 3 & 43.500 & 44.500 & 43.500 & 43.500 & 43.500 & 43.500 & 43.501 & 43.500\\
3 & 4 & 54.500 & 54.500 & 54.500 & 54.500 & 54.500 & 54.500 & 54.500 & 54.500\\
4 & 5 & 62.500 & 62.501 & 62.501 & 62.503 & 62.501 & 62.500 & 62.500 & 62.500\\
\bottomrule
\end{longtable}
\endgroup{}

\newpage

\section{Key Findings}\label{key-findings}

\begin{enumerate}
\def\labelenumi{\arabic{enumi}.}
\item
  \textbf{Score precision}: Skalapoeng values have 3-decimal precision
\item
  \textbf{Clean separation}: All level boundaries have positive gaps (no
  misclassifications)
\item
  \textbf{Threshold format}: Most thresholds are at X.5 (e.g., 42.5,
  56.5)
\item
  \textbf{Notable threshold changes}:

  \begin{itemize}
  \tightlist
  \item
    \textbf{NOR Grade 5 Level 2→3}: Changed from 56.5 to 57.5 starting
    2016
  \item
    \textbf{NOR Grade 8 Level 1→2}: Changed from 36.5 to 37.5 starting
    2016
  \item
    \textbf{NOR Grade 8 Level 2→3}: Was 44.5 in 2015, otherwise 43.5
  \item
    \textbf{MATH Grade 8 in 2014}: Used integer thresholds (37, 45, 55,
    63)
  \end{itemize}
\item
  \textbf{Percentile alignment}: Actual thresholds generally align well
  with target percentiles
\end{enumerate}

\section{Interpretation}\label{interpretation}

\subsection{How to Use These
Thresholds}\label{how-to-use-these-thresholds}

\begin{verbatim}
For Grade 5 (3 levels):
  Level 1: score < threshold_1
  Level 2: threshold_1 <= score < threshold_2
  Level 3: score >= threshold_2

For Grade 8 (5 levels):
  Level 1: score < threshold_1
  Level 2: threshold_1 <= score < threshold_2
  Level 3: threshold_2 <= score < threshold_3
  Level 4: threshold_3 <= score < threshold_4
  Level 5: score >= threshold_4
\end{verbatim}

\subsection{Comparison to Published Udir
Thresholds}\label{comparison-to-published-udir-thresholds}

The published Udir thresholds are integers (e.g., 43, 57). The empirical
thresholds are typically 0.5 below the published values (e.g., 42.5,
56.5). This is because Udir rounds up for communication, while actual
classification uses the precise decimal cutpoint.

\end{document}
